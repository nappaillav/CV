%-------------------------
% Resume in Latex
% Author : Sourabh Bajaj
% License : MIT
%------------------------

\documentclass[letterpaper,10.5pt]{article}
%\usepackage{times}

\usepackage{latexsym}
\usepackage[empty]{fullpage}
\usepackage{titlesec}
\usepackage{marvosym}
\usepackage[usenames,dvipsnames]{color}
\usepackage{verbatim}
\usepackage{enumitem}
\usepackage[pdftex]{hyperref}
\usepackage{fancyhdr}
\makeatletter
\newcommand*{\rom}[1]{\expandafter\@slowromancap\romannumeral #1@}
\makeatother

\pagestyle{fancy}
\fancyhf{} % clear all header and footer fields
\fancyfoot{}
\renewcommand{\headrulewidth}{0pt}
\renewcommand{\footrulewidth}{0pt}

% Adjust margins
\addtolength{\oddsidemargin}{-0.36in}
\addtolength{\evensidemargin}{-0.36in}
\addtolength{\textwidth}{1in}
\addtolength{\topmargin}{-.50in}
\addtolength{\textheight}{0.75in}

\urlstyle{same}


\raggedbottom
\raggedright
\setlength{\tabcolsep}{0in}

% Sections formatting
\titleformat{\section}{
  \vspace{-4pt}\scshape\raggedright\large
}{}{0em}{}[\color{black}\titlerule \vspace{-5pt}]

%-------------------------
% Custom commands
\newcommand{\resumeItem}[2]{
  \item\small{
    \textbf{#1}{: #2 \vspace{-2pt}}
  }
}

\newcommand{\resumeSubheading}[4]{
  \vspace{-1pt}\item
    \begin{tabular*}{0.97\textwidth}{l@{\extracolsep{\fill}}r}
      \textbf{#1} & #2 \\
      \textit{\small#3} & \textit{\small #4} \\
    \end{tabular*}\vspace{-4pt}
}

\newcommand{\resumeSubItem}[2]{\resumeItem{#1}{#2}\vspace{-2pt}}

\renewcommand{\labelitemii}{$\circ$}

\newcommand{\resumeSubHeadingListStart}{\begin{itemize}[leftmargin=*]}
\newcommand{\resumeSubHeadingListEnd}{\end{itemize}}
\newcommand{\resumeItemListStart}{\begin{itemize}}
\newcommand{\resumeItemListEnd}{\end{itemize}\vspace{-4pt}}

%-------------------------------------------
%%%%%%  CV STARTS HERE  %%%%%%%%%%%%%%%%%%%%%%%%%%%%


\begin{document}

%----------HEADING-----------------
\begin{tabular*}{\textwidth}{l@{\extracolsep{\fill}}r}
  \textbf{\href{https://prrvalli.wixsite.com/valliappan-ca}{\Large \textcolor{blue}{Valliappan CA}}} & Link to my \href{https://scholar.google.co.in/citations?user=5aAnMwoAAAAJ&hl=en}{\textcolor{blue}{Google-Scholar}} $|$ Github ID : \href{https://github.com/nappaillav}{\textcolor{blue}{nappaillav}}\\
  Email : \href{mailto:valliappan@paralleldots.com}{valliappan@paralleldots.com}  \\%Github ID:\href{https://github.com/PRRvalli}{PRRvalli} Email : \href{mailto:spmanikam@gmail.com}{spmanikam@gmail.com}
\end{tabular*}
%----------Research interest-----------------
\section{Research Interest}
  
      {Deep Learning application in Speech Signal Processing and Computer Vision}
      \vspace*{-2.5mm}
%-----------EDUCATION-----------------
\section{Education}
  \resumeSubHeadingListStart
    \resumeSubheading
      {Birla Institute of Technology and Science, Pilani}{India}
      {Bachelor of Engineering in Electronics and Instrumentation;  GPA: 8.03/10}{Aug 2013 - Aug 2017}
      \begin{comment}
    \resumeSubheading
    {Maharishi Vidya Mandir}{Chennai, India}
      {Class \rom{12}:Central Board of Secondary Education  Percentage: 93.2\% }{ Aug 2012 - July 2013}
      \resumeSubheading
    {Maharishi Vidya Mandir}{Chennai, India}
     {Class \rom{10}:Central Board of Secondary Education  Percentage: 98\%}{ Aug 2010 - July 2011}
      \end{comment}
  \resumeSubHeadingListEnd
  \vspace*{-2.5mm}
%-----------Research Experience-----------------
\section{Bachelor Thesis}
\resumeSubHeadingListStart

    \resumeSubheading
      {Study of Broadband Reflectometry Data Using Non-Stationary Signal Processing}{BITS Pilani, India}
      {Guide: Dr.\href{http://universe.bits-pilani.ac.in/goa/amalinprince/profile}{\textcolor{blue}{A.Amalin Prince}}}{Jan 2017 - May 2017}
      \resumeItemListStart
        \resumeItem{Abstract}
          {Presented a comparative study of linear and non-linear signal processing methods
like Wavelet decomposition, Empirical Mode Decomposition \& Ensemble Empirical Mode
Decomposition on Broadband reflectometry data.\href{https://drive.google.com/file/d/10t6c10gRrJtQz1KwfsXgxr7PS_ABksOd/view?usp=sharing}{\textcolor{blue}{[PDF]}}}
      \resumeItemListEnd

 \resumeSubHeadingListEnd
 \vspace*{-2.5mm}
%-----------Publication-----------------
\section{Publication}
 \begin{itemize}
 \item Renuka Mannem, \textbf{C.A.Valliappan}, Prasanta Kumar Ghosh.``A SegNet Based Image Enhancement Technique for Air-Tissue Boundary Segmentation in Real-Time Magnetic Resonance Imaging Video'', accepted in $25^{th}$ National Conference on Communications (NCC), 2019.
 \item \textbf{C.A.Valliappan}, Renuka Mannem, Prasanta Kumar Ghosh. ``Air-Tissue Boundary Segmentation in Real-Time Magnetic Resonance Imaging Video using Semantic Segmentation with Fully Convolutional Networks'', accepted in  Interspeech 2018.\href{https://www.isca-speech.org/archive/Interspeech_2018/pdfs/1939.pdf}{\textcolor{blue}{[Link]}}
 \item \textbf{C.A.Valliappan}, Anurag Das, Prasanta Kumar Ghosh. ``Classification of story-telling and poem recitation using head gesture of the talker'', accepted in  International Conference on Signal Processing and Communications (SPCOM) 2018. \href{https://drive.google.com/open?id=1UUibe_Zs8rFzP__e8I4D4Wb0as40ep4d}{\textcolor{blue}{[Link]}}
	\item Nalband S, \textbf{Valliappan CA}, Prince AA, \& Agarwal, ``Time frequency based feature extraction for the analysis of vibroarthographic signals'', in journal of Computers and Electrical Engineering, Elsevier Publication, 2018. \href{https://www.sciencedirect.com/science/article/pii/S0045790617329087}{\textcolor{blue}{[Link]}}
	\item Nalband S, \textbf{Valliappan CA}, Gupta R, Prince AA, and Agarwal A, ``Feature Extraction and Classification of Knee Joint Disorders Using Hilbert Huang Transform'', $14^{th}$ IEEE Conference of ECTI Society, (ECTI-CON), 2017. \href{http://ieeexplore.ieee.org/abstract/document/8096224/}{\textcolor{blue}{[Link]}}
	\item Balaji A, Haldar A, Patil K, Ruthvik TS,\textbf{ C.A. Valliappan}, Jartarkar M, Baths V. ``EEG-based classification of bilingual unspoken speech using ANN'', in Engineering in Medicine and Biology Society (EMBC), pp. 1022-1025, 2017.  \href{http://ieeexplore.ieee.org/document/8037000/}{\textcolor{blue}{[Link]}}
	\item \textbf{C.A. Valliappan}, Advait Balaji, Sai Ruthvik Thandayam, Piyush Dhingra, and Veeky Baths. `` A Portable Real Time ECG Device for Arrhythmia Detection Using Raspberry Pi'', in International Conference on Wireless Mobile Communication and Healthcare, pp. 177-184. Springer, Cham, 2016. \href{https://www.springerprofessional.de/en/a-portable-real-time-ecg-device-for-arrhythmia-detection-using-r/12336602}{\textcolor{blue}{[Link]}}
	

\end{itemize} 
%-----------Research Experience-----------------
\section{Work Experience}
\resumeSubHeadingListStart
\resumeSubheading
      {Data Scientist at ParallelDots}{Gurgaon, India}
      {Mentor:\href{https://www.linkedin.com/in/muktabh}{\textcolor{blue}{Muktabh Mayank}}}{October  2018 - Present}
      \resumeItemListStart
          \resumeItem{Role}
          {Object tracking and Optical Character Recognition used for shelf monitoring}
		\begin{itemize}
		\item To develop \textbf{Automated Retail Auditing Tool }using state of the art object tracking techniques like SSD, YOLO and Faster RCNN. ‘ShelfWatch-Demo’ \href{https://www.karna.ai/retail-shelf-monitoring}{\textcolor{blue}{Link}}
		\item Generation of artificial dataset from few images of retail products. 
		\item Implemention of the Adaptive sampling on the dataset for improving the learning of the model.  
		%\item Identification of duplicate image for shop monitoring using key point matching algorithm '\textbf{SuperPoint}'.
		\item Analyzing the intricate details of the model, to reduce the false positives. 
		\item OCR is used to supplement the results with addition details from the tracked objects. Custom trained SSD for text detection
			
		\end{itemize}    
      \resumeItemListEnd

    \resumeSubheading
      {Research Assistant at SPIRE lab}{IISc, Bangalore, India}
      {Mentor: Dr.\href{http://www.ee.iisc.ac.in/new/people/faculty/prasantg/}{\textcolor{blue}{Prasanta Kumar Ghosh}}}{June  2017 - September 2018}
      % {Guide:Dr.\href{http://www.ee.iisc.ac.in/new/people/faculty/prasantg/}{Prasanta Kumar Ghosh}}
      \resumeItemListStart
        %\resumeItem{Guide}
          %{Dr.\href{http://www.ee.iisc.ac.in/new/people/faculty/prasantg/}{Prasanta Kumar Ghosh}}
          \resumeItem{Topic}
          {Analysis and Synthesis of Head motion for Human-Computer Interaction}
		\begin{itemize}
		\item 3D-Audio visual Head motion database was created using Motion Capture System(\href{http://spire.ee.iisc.ac.in/spire/mocap.php}{MoCap}).
		\item Estimating the Translation coordinates \& Euler angle from the head trajectory over the entire database.
\item Investigate the nature of head gestures in spontaneous speech in comparison to that in rhythmic speech(singing).	
\end{itemize}

\resumeItem{Topic}
          {Audio-Visual Synthesis for Realistic Agent}
		

\begin{itemize}


	\item  Extraction of Lip shape from Audio Visual Corpus.\href{http://spire.ee.iisc.ac.in/spire/database.php}{\textbf{PRAV}} \item  Synthesis of Lip movement from the acoustic feature of the speaker using \textbf{LSTM} network. 
		\item Optimal loss lip motion synthesized using Dynamic programming.  \href{https://www.youtube.com/watch?v=rBgG4szMQhQ}{Video}
			
		\end{itemize}    
      \resumeItemListEnd
      \begin{comment}
      \resumeSubheading
      {Intern at SPIRE lab}{IISc, Bangalore, India}
      {Guide:Prasanta Kumar Ghosh}{May 2016 - Aug 2016}
      
      \resumeItemListStart
        %\resumeItem{Guide}
          %{Dr.\href{http://www.ee.iisc.ac.in/new/people/faculty/prasantg/}{Prasanta Kumar Ghosh}}
          \resumeItem{Topic}
          {Audio-Visual Synthesis for Realistic Agent}
		\begin{itemize}
		\item  Extraction of Lip shape from Audio Visual Corpus.\href{http://spire.ee.iisc.ac.in/spire/database.php}{\textbf{PRAV}} \item  Synthesis of Lip movement from the acoustic feature of the speaker using \textbf{LSTM} network. 
		\item Optimal loss lip motion synthesized using Dynamic programming. Demo  \href{https://www.youtube.com/watch?v=rBgG4szMQhQ}{Video}    
		\end{itemize}    
      \resumeItemListEnd
      
      \resumeSubheading
      {Intern at Madras Atomic Power Station}{Kalpakkam, India}
      {Guide:N.Chokalingam}{May 2015 - July 2015}
      
      \resumeItemListStart
        %\resumeItem{Guide}
          %{Dr.\href{http://www.ee.iisc.ac.in/new/people/faculty/prasantg/}{Prasanta Kumar Ghosh}}
          \resumeItem{Topic}
          {Amplifier Design For Area Radiation Monitoring System}
		\begin{itemize}
		\item Reviewing \& modifying the amplifier system in the Radiation Monitoring unit
of the plant with industrial standard ICs.
		\end{itemize}    
      \resumeItemListEnd
      \end{comment}

 \resumeSubHeadingListEnd

%-----------Projects-----------------
\section{Projects}
\resumeSubHeadingListStart

\resumeSubheading
      {Decision making system using the reviewer scores and comments for the submitted paper}{IISc Bangalore}
      {Guide:Prof.Prasanta Kumar Ghosh}{August 2018}
		
		\vspace*{-1.5mm}
		\begin{itemize}
		\item Xgboot model was learnt using the features reviewer comment and scoces. \href{https://github.com/nappaillav/reviewclassification}{\textcolor{blue}{[code]}}\vspace*{-1.5mm}
		\item The model was trained on the previous years data and tested on current edition, which gave an accuracy of $\sim90\%$.\vspace*{-1.5mm}
		\end{itemize}

\resumeSubheading
      {Air-Tissue Boundary segmentation in rt-MRI video using semantic segmentation}{IISc Bangalore}
      {Guide:Prof.Prasanta Kumar Ghosh}{Dec 2017 - March 2018}
		
		\vspace*{-1.5mm}
		\begin{itemize}
		\item Semantic segmentation using the Deep learning architecture called Fully Convolutional Networks and SegNet.\vspace*{-1.5mm}
		\item Currently, my approach to this ATB segmentation produces $\sim10\%$ less error rate than the baseline. \href{https://github.com/nappaillav/RTMRI-Segmentation}{\textcolor{blue}{[code]}}\vspace*{-1.5mm}
		\end{itemize}    

    
    \resumeSubheading
      {De-Noising technique for frequency signal recorded from ships}{BITS Pilani}
      {Guide:Prof.Neena Goveas}{Jan 2017 - May 2017}
		
		\vspace*{-1.5mm}
		\begin{itemize}
		\item This project involved experimenting different sequential models for denoising the frequency signals.\vspace*{-1.5mm}
		\item Experimented various loss functions other than MSE for deployed model.\vspace*{-1.5mm}
		\end{itemize}    
 \begin{comment}   
      
%      \resumeSubheading
%      {Audio-Visual Synthesis}{IISc Bangalore, India}
%      {Guide:Prof. Prasanta Kumar Ghosh}{Jun 2016 - Aug 2016}
%		\begin{itemize}
%		\item  Extraction of Lip shape from Audio Visual Corpus.\href{http://spire.ee.iisc.ac.in/spire/database.php}{\textbf{PRAV}} \item  Synthesis of Lip movement from the acoustic feature of the speaker using \textbf{LSTM} network.
%		\end{itemize}  
		
	
      
\resumeSubheading
	% polish it with a better title 
      {EEG-based Classification of Bilingual Unspoken Speech using ANN}{BITS Pilani}
      {Guide:Prof.Veeky Baths}{Dec 2016 - Feb 2017}
     
\vspace*{-1.5mm}
		\begin{itemize}
		\item This work was done to understand the ability of EEG signal to interpret unspoken or imagined speech.\href{https://github.com/nappaillav/Heart_Rate_monitoring_Device}{\textcolor{blue}{[code]}}\vspace*{-1.5mm}
		%\item Find active regions of brain towards different language \& decision making.
		\item The best accuracy of 85.20\% \& 92.18\% for decision and language classification respectively and the overall accuracy of bilingual speech classification was 75.38\%.
		\vspace*{-1.5mm}
		\end{itemize}
\end{comment}
     \resumeSubheading
	% polish it with a better title 
      {Comparative study of signal decomposition technique using Time-Frequency Image}{BITS Pilani}
      {Guide:Prof.A. Amalin Prince}{Jan 2016 - Dec 2016}
     \vspace*{-1.5mm}

		\begin{itemize}
		\item Investigated the non-linear decomposition techniques like Wavelet, Empirical Mode Decomposition(EMD) \& Ensemble Empirical Mode Decomposition(EEMD) for time-frequency imaging.\vspace*{-1.5mm}
\item Analyzed the performance of each method using the features extracted from Time Frequency image. \vspace*{-1.5mm}

		\end{itemize}		
		
\resumeSubheading
	% polish it with a better title 
      {Portable Real Time ECG Device for Arrhythmia Detection Using Raspberry Pi}{BITS Pilani}
      {Guide:Prof.Veeky Baths}{Jan 2016 - Dec 2016}
      
\vspace*{-1.5mm}
		\begin{itemize}
	    	\item Real-Time ECG tracking device with special emphasis on arrhythmia detection, completely Portable with Mobile Application, Build using RasPi-3.\href{https://github.com/nappaillav/Heart_Rate_monitoring_Device}{\textcolor{blue}{[code]}}
	    	 \vspace*{-1.5mm}
	    	%\item Making it completely Portable with Mobile Application, Build using RaspPi-3.
	    	\item The arrhythmia detection algorithm tested on MIT-BIH database and reported an accuracy greater than 95\%.
	    	\vspace*{-1.5mm}
		\end{itemize}
		
\begin{comment}		
\resumeSubheading
	% polish it with a better title 
      {Feature Extraction and Classification of Knee Joint signal}{BITS Pilani}
      {Guide:Prof.A. Amalin Prince}{Aug 2015 - Dec 2015}\vspace*{-1.5mm}
		\begin{itemize}
		\item Hilbert Huang Transform was applied on the signal to extract the features.\vspace*{-1.5mm}
		\item Performance was evaluated and compared to the previously existing methods using LS-SVM \& Random Forest.\vspace*{-1.5mm}
		\end{itemize}	  
 \end{comment}   		
 \resumeSubHeadingListEnd
%-----------Public repository-----------------
%\section{Public Repository}

 %      \textbf{Phoneme to Word (Phoneme2word)}
  %     \vspace*{-2mm}
   %   \begin{itemize}
    %      \item Phoneme2word is a tool to convert phonemes sequences from forced alignment results into words sequence. Link to the \href{https://github.com/PRRvalli/phoneme2word}{Repository}\vspace*{-2mm}
     % \end{itemize}
    %\end{itemize}  

%-----------Position of Responsibility-----------------
\section{Position of Responsibility}

\resumeSubHeadingListStart
\resumeSubheading
      {Teaching Assistant for EEE-F434 Digital Signal Processing}{BITS Pilani}
      {Set programming assignments for Hardware and Software implementation
of DSP algorithms}{Jan 2016-May 2016}
      \resumeSubheading
      {Teaching Assistant for CS-F111 Computer Programming}{BITS Pilani}
      {Evaluated assignments of the students and ensured smooth functioning of
Lab session}{Jan 2015-May 2015}
\resumeSubHeadingListEnd




%-----------Skills-----------------

\section{Skills}
\begin{itemize}
\item \textbf{Languages}: C, C++, Python, MATLAB, UNIX Shell Script, \LaTeX.
\item \textbf{Deep Learning Libraries}: Pytorch, Keras and Tensorflow.
\end{itemize}
%-----------Achievements-----------------
\section{Achievements \& Awards}
\begin{itemize}
\item Travel Grant Recipient for \href{https://interspeech2018.org/}{\textcolor{blue}{INTERSPEECH}} - 2018. 
\item Participated at \href{https://icpc.baylor.edu/ICPCID/QDI825ZM40CC}{\textcolor{blue}{ACM-ICPC}} Asia Amritapuri Multisite Regional Contest 2016 and 2014.
\item Course topper in \textbf{Digital Signal Processing , Digital design}
\item Awarded the best Student project in \href{https://hackinout.co/}{Hackathon INOUT 3.0} held in NIT Surat for "The Portable Real Time ECG Device using Raspberry Pi". This work was presented as a paper in \href{http://archive.mobihealth.name/2016/show/home}{\textcolor{blue}{Mobhihealth}}-2016, in Milan.

\end{itemize}
%-----------EXPERIENCE-----------------
\footnote{Last Updated: \today}
%-------------------------------------------
\end{document}
